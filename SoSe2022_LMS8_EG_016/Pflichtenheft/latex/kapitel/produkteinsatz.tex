

\begin{markdown}

## Zielgruppen

Das Produkt soll für AuA jeden Alters ohne weiterführende (technische) Vorkenntnisse - die über die Verwendung eines Appfähigen Geräts, die Installation einer App oder Bedienung eines Web-Browsers hinausführen - geeignet sein.

Da AuA Teil der Projektergebnisse sein könnten, wird eine hohe Partizipation in der Webanwendung und App erwartet. Diese Beteiligung ermöglicht dem Projektleiter zusätzliche Informationen der potentiellen Anwenderinnen und Anwender einzuholen, welche sich potentiell positiv auf die Qualität der Projekte auswirken können.

Neben den geringfügigen technischen Vorkenntnissen, sollen die AuA jedwige Information über Projekte problemlos finden. Daher wird auch sichergestellt werden, dass die Bedienung der Webanwendung und App leicht verständlich ist, damit die AuA die Nutzung der Webanwendung und App maximieren und eine Vielzahl von etwaigen wichtigen aber noch unbekannten Informationen mitteilen.

In Achtung des digitalen Zeitgeistes wird eine intensivere Nutzung der App erwartet. Infolgedessen kann die Endfassung der App mehr Funktionen bereitstellen als die Webanwendung. AuA werden auf ihren Wegen im Alltag mobile Geräte benutzen und sollen dabei durch eine schneller und unkomplizitere Öffnung der App (im Vergleich zur Webanwendung) unterstützt werden. Dennoch sollen AuA mit geringfügiger Präferenz appfähige Geräte zu benutzen die Möglichkeit erhalten Projektinformationen einzusehen.
\end{markdown}
\newpage

\begin{markdown}
## Anwendungsgebiete

AuA in den Planungs- und Umsetzungsprozess eines Projekts integrieren. "Integrieren" der AuA in jeden Prozess ist dabei wie folgt definiert:

1. Aktuelle Informationen zum entsprechenden Projekt und zugehörigen Teilprojekten bereitstellen. Transparenz des Projektvorgangs soll die Außenwirkung des Projekts positiv beeinflussen und die Anzahl von AuA erhöhen.
2. Soziale Involvierung von AuA durch individuelle Meinungsverbreitung in Form von Kommentaren zu (Teil-)Projekten. AuA sollen angeregt werden über Teile von Projekten zu diskutieren und so auch eine umfassendere und reflektierte Meinung zu entsprechenden Themen aufzubauen, die den Projektprozess unterstützen und die Qualität des Endprodukt des Projekts steigern könnten.
3. Alle Projektinformationen werden dauerhaft verfügbar sein, um eine barrierefreie Nutzung der App oder Webanwendung zu gewährleisten. Somit sind Projektinformationen zu jeder Zeit und an jedem Ort spontan verfügbar.
\end{markdown}
