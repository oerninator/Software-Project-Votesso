\section{Komponentendiagramm Web}

\begin{figure}[H]
	\centering
	\includegraphics[width=\textwidth]{img/componentweb.png}	
	\caption{Komponentendiagramm - Web}
	\label{fig:komponentendiagramm-a}
\end{figure}

\begin{table}[H]
	\centering
	\begin{tabularx}{\textwidth}{X X}
		\rowcolor[HTML]{C0C0C0} 
		\textbf{Subkomponente} & \textbf{Aufgabe} \\
		HTML-handler & Verwaltet und erstellt HTML-Dokumente und deren Inhalt. \\
		\rowcolor[HTML]{E7E7E7}
		project management & Ruft Daten für Projekte auf und stellt sie dem jeweiligen HTML-Dokument bereit.   \\
		 subproject management &  Ruft Daten für Teilprojekte auf und stellt sie dem jeweiligen HTML-Dokument bereit. \\
		 \rowcolor[HTML]{E7E7E7}
		comment management & Ruft Daten für Kommentare auf und stellt sie dem jeweiligen HTML-Dokument (Detailübersicht eines Teilprojekts) bereit. Verwaltet des Weiteren den Löschvorgang von Kommentaren für Administratoren. \\
		admin management & Ruft Daten für die Anmeldung als Administrator auf und stellt sie dem jeweiligen HTML-Dokument bereit. Verwaltet außerdem die Autorisierung zum Löschen von Kommentaren.   \\
		\rowcolor[HTML]{E7E7E7}
		repositories & Stellt den angefragten Datenbestand der PostgreSQL-Datenbank bereit und ist zentral für die nachträgliche Aktualisierung der Daten im Falle einer Änderung (z.B. Kommentar erstellen oder löschen). \\
		API-services & Verwaltet den Datenfluss zwischen Backend und App. \\
		
	\end{tabularx}
	\caption{Komponentenbeschreibung - Web}
	\label{table:komponentenbeschreibung-web}
\end{table}


\section{Komponentendiagramm App}

\begin{figure}[H]
	\centering
	\includegraphics[width=\textwidth]{img/componentapp.png}	
	\caption{Komponentendiagramm - App}
	\label{fig:komponentendiagramm-b}
\end{figure}

\begin{table}[H]
	\centering
	\begin{tabularx}{\textwidth}{X X}
		\rowcolor[HTML]{C0C0C0} 
		\textbf{Subkomponente} & \textbf{Aufgabe} \\
		write and show comments & Diese Komponente ermöglicht es dem User zu vorher ausgewählten Teilprojekten einen für alle anderen User sichtbaren Kommentar abzugeben oder Kommentare von anderen Usern anzuschauen. \\
		\rowcolor[HTML]{E7E7E7}
		create and show evaluation & Diese Komponente bietet dem User die Möglichkeit eine Bewertung in Form von Daumen hoch oder Daumen runter abzugeben und die Bewertung anderer User anzuschauen.   \\
		 Repository &  Diese Komponente speichert die benötigten Daten zur Laufzeit und stellt sie den anderen Komponenten zur Verfügung. Außerdem kommuniziert sie mit dem APIService um Daten zu senden und zu empfangen. \\
		 \rowcolor[HTML]{E7E7E7}
		show projects & In dieser Komponente werden dem User alle verfügbaren Hauptprojekte angezeigt. Außerdem ist sie dafür verantwortlich nach Auswahl eines Hauptprojektes die entsprechende Detailseite anzuzeigen. \\
		show subprojects & In dieser Komponente werden dem User alle zu einem Hauptprojekt zugehörigen Teilprojekte,  nach Entfernung sortiert, angezeigt. Zudem sorgt diese Komponente auch dafür, die entsprechende Teilprojekt Detailseite anzuzeigen.   \\
		\rowcolor[HTML]{E7E7E7}
		APIService & Diese Komponente übernimmt die Kommunikation mit dem Backend. Sie ist sowohl dafür verantwortlich beim Start der App die Daten zu holen, als auch neue Daten zu senden. \\
		API-services & Verwaltet den Datenfluss zwischen Backend und App. \\
		
	\end{tabularx}
	\caption{Komponentenbeschreibung - App}
	\label{table:komponentenbeschreibung-app}
\end{table}